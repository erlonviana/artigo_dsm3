%%%% fatec-article.tex, 2024/03/10

%% Classe de documento
\documentclass[
  landscape,
  a4paper,
  10pt, %% Tamanho de fonte reduzido
  english,
  brazilian,
]{article}

%% Pacotes utilizados
\usepackage[landscape, margin=0.5in]{geometry} %% Margens reduzidas
\usepackage[]{fatec-article}
\usepackage{setspace}
\usepackage{graphicx}
\usepackage{array}
\usepackage{longtable}

%% Início do documento
\begin{document}

\section*{Diário de Bordo}%
\begin{longtable}{|>{\centering\arraybackslash}p{3.5cm}|>{\centering\arraybackslash}p{2.5cm}|>{\centering\arraybackslash}p{2.5cm}|>{\centering\arraybackslash}p{3.5cm}|>{\arraybackslash}p{6cm}|}
\hline
Nome da Atividade & Data de início & Data de término & Responsável pela atividade & Descrição da atividade realizada \\ \hline
Definição da equipe & 06/07 & 06/07 & Todos & Definição dos integrantes do grupo \\ \hline
Reunião & 11/07 & 11/07 & Todos & Reunião para revisão do tema e distribuição das atividades \\ \hline
Protótipo da fazenda & 09/08 & 31/10 & Érlon & Desenvolvimento de um protótipo pequena escala da fazenda vertical \\ \hline
Banco de dados & 01/09 & 19/09 & Érlon & Elaboração de um banco de dados não-relacional \\ \hline
Backend & 01/09 & 19/09 & Érlon & Desenvolvimento de um backend utilizando JavaScript/Node \\ \hline
Artigo & 01/09 & 31/10 & Érlon & Revisão do artigo existente (Introdução, objetivo e estado da arte) \\ \hline
Documentação do banco de dados & 01/09 & 19/09 & Érlon & Elaborar, juntamente ao banco de dados, a documentação referente \\ \hline
Figma & 01/09 & 14/09 & Leandro & Desenvolver uma interface UX/UI conforme orientações da equipe \\ \hline
Reunião & 13/09 & 13/09 & Todos & Reunião para acompanhamento e distribuição das tarefas \\ \hline
Scrum & 14/09 & 14/09 & Todos & Elaboração de um SCRUM para visualização das tarefas \\ \hline
Site da equipe & 14/09 & 27/10 & Andrei & Revisar o site da equipe \\ \hline
Reunião & 14/10 & 14/10 & Todos & Reunião de alinhamento \\ \hline
Início do Front-end & 14/10 & 14/10 & Ricardo & Iniciando a criação da side-bar \\ \hline
Desenvolvimento da sidebar & 14/10 & 15/10 & Ricardo & Finalização da sidebar e início da criação do dashboard \\ \hline
Dashboard & 15/10 & 16/10 & Ricardo & Criação do dashboard sem implementar o banco de dados mockados \\ \hline
Conexão com API & 16/10 & 16/10 & Ricardo & Criação da conexão com a API \\ \hline
Canvas & 16/10 & 16/10 & Érlon & Revisar o canvas (modelo do Sebrae) \\ \hline
Diagrama de objetos & 16/10 & 26/10 & Érlon & Desenvolver o diagrama de objetos do projeto \\ \hline
Início do CRUD & 17/10 & 17/10 & Ricardo & Iniciando as telas do CRUD no front-end \\ \hline
Conclusão do CRUD & 17/10 & 22/10 & Ricardo & Finalização das telas do CRUD funcional \\ \hline
Diagrama de classes & 20/10 & 20/10 & Érlon & Elaboração do diagrama de classes \\ \hline
Pesquisa de usuário & 20/10 & 24/10 & Todos & Pesquisa on-line para identificação de perfis de usuário e oportunidades do projeto \\ \hline
Swot & 21/10 & 21/10 & Érlon & Desenvolver matriz SWOT \\ \hline
Diagrama e Especificações da Infraestrutura da rede & 22/10 & 22/10 & Érlon & Elaborar diagrama de redes \\ \hline
Ajustes de responsividade & 23/10 & 23/10 & Ricardo & Ajustes de responsividade e finalização das telas do projeto \\ \hline
Diagrama de casos de uso & 23/10 & 23/10 & Érlon & Elaborar diagrama de casos de uso \\ \hline
Tela de login & 24/10 & 26/10 & Andrei & Criação da tela de login estilizada e responsiva \\ \hline
Pitch & 24/10 & 24/10 & Ricardo & Vídeo de 1 min com legendas em português + inglês \\ \hline
Ajustes na tela de login & 27/10 & 27/10 & Andrei & Ajuste nas cores da tela de login \\ \hline
Revisão final & 28/10 & 28/10 & Todos & Revisão do funcionamento do front-end e back-end \\ \hline
Entrega do PI & 29/10 & 29/10 & Érlon & Submissão do Projeto Integrador no TEAMS \\ \hline
\end{longtable}

\end{document}
