%%%% fatec-article.tex, 2024/03/10

\documentclass[
  a4paper,%% Tamanho de papel: a4paper, letterpaper (^), etc.
  12pt,%% Tamanho de fonte: 10pt (^), 11pt, 12pt, etc.
  english,%% Idioma secundário (penúltimo) (>)
  brazilian,%% Idioma primário (último) (>)
]{article}

%% Pacotes utilizados
\usepackage[]{fatec-article}

%% Início do documento
\begin{document}
\vspace{8cm}
\begin{center}
    \large \textbf{\title{ARTEFATOS DO PROJETO DE SOFTWARE}}
\end{center}

\maketitle

\break

\tableofcontents

\break


%exemplo da forma de organização das seções e subseções, você deverá adaptar o template para a realidade do seu projeto.

\section*{Diagramas UML}
    Nesta seção serão apresentados os diagramas da UML utilizados para a modelagem do sistema desenvolvido. No nosso caso foi desenvolvido o diagrama de Caso de Uso.
    
    \subsection*{Diagrama de Caso de Uso}
    \addcontentsline{toc}{section}{Diagrama de Caso de Uso}

    Esse é um exemplo de diagrama de caso de uso, você deverá descrever todos os componentes presentes no diagrama (atores e funcionalidades do sistema).

            \begin{figure}[h]
\centering
\caption{Diagrama de caso de uso}
\label{fig:diagrama-caso-uso}
 \includegraphics[scale=0.5]{Logos/Diagrama em branco.png}
\SourceOrNote{Maia (2024)}
\end{figure}

    No contexto da Fazenda Vertical orientada por Redes Neurais foram mapeadas as principais necessidades do usuário. Desta forma, identificamos 9 necessidades básicas (registro, login, tela inicial, escolher tipo de hortaliça, escolha do fertilizante, ver o dashboard, definir o nível de automação, configurar dados e ver câmeras em tempo real) que são atividades cruciais para a experiência do usuário e o sucesso do projeto.
    

\end{document}