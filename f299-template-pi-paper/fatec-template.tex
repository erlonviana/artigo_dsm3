%%%% fatec-article.tex, 2024/03/10

%% Classe de documento
\documentclass[
  a4paper,%% Tamanho de papel: a4paper, letterpaper (^), etc.
  12pt,%% Tamanho de fonte: 10pt (^), 11pt, 12pt, etc.
  english,%% Idioma secundário (penúltimo) (>)
  brazilian,%% Idioma primário (último) (>)
]{article}

%% Pacotes utilizados
\usepackage[]{fatec-article}
\Author{1}{Name={Araujo, A\\ Santos, E \\ Sueoka, L \\ Estevam, R }}

\Author{2}{Name={\{ andrei.araujo01@fatec.sp.gov.br \}\\ \{ erlon.santos3@fatec.sp.gov.br \} \\ \{ leandro.sueoka@fatec.sp.gov.br \} \\ \{ ricardo.conceicao@fatec.sp.gov.br \} }}

%% Definição das palavras-chaves/keywords
\Keyword{1}{Fazendas verticais}{Vertical farms}
\Keyword{2}{Inteligencia artificial}{Artificial intelligence}
\Keyword{3}{ODS}{SDG}

%%%% Resumo no idioma primário (brazilian)
\begin{Abstract}[brazilian]%% Idioma (brazilian ou english)
  Fome, agricultura sustentável, indústria, inovação e agricultura sustentável são alguns dos desafios enfrentados no Brasil e no mundo apontados pela Organização das Unidas. Neste sentido o conceito de fazendas verticais vem se destacando, porém necessita de pesquisas para maior produtividade, redução de custos e perdas. Este estudo propõe o desenvolvimento de um sistema de gerenciamento de fazendas verticais orientado por inteligência artificial. A fazenda terá sensores de fluxo de água e concentração de fertilizante que terão seus dados enviados para um servidor na nuvem. Os dados serão avaliados por uma inteligência artificial que analisará as informações, realizando os ajustes necessários para a fazenda vertical. A metodologia utilizada consiste na prototipação de interfaces, diagramação de banco de dados e do sistema, análise de projeto (via Canvas) e criação de uma plataforma web low-code (Apex). Os resultados mostram que é possível o desenvolvimento de um sistema com estas características, melhor produtividade, menos perdas, menor consumo de água, eletricidade e fertilizante utilizados em uma fazenda vertical.
\end{Abstract}

%%%% Resumo no idioma secundário (english)
\begin{Abstract}[english]%% Idioma (brazilian ou english)
  Hunger, sustainable agriculture, industry, innovation and sustainable agriculture are some of the challenges faced in Brazil and the world highlighted by the United Nations. In this sense, the concept of vertical farms has been gaining prominence, but it requires research to achieve greater productivity, reduce costs and losses. This study proposes the development of an artificial intelligence-driven vertical farm management system. The farm will have water flow and fertilizer concentration sensors that will have their data sent to a cloud server. The data will be evaluated by artificial intelligence that will analyze the information, making the necessary adjustments to the vertical farm. The methodology used consists of interface prototyping, database and system diagramming, project analysis (via Canvas) and creation of a low-code web platform (Apex). The results show that it is possible to develop a system with these characteristics, better productivity, less losses, lower consumption of water, electricity and fertilizer used in a vertical farm.
\end{Abstract}

%% Processamento de entradas (itens) do índice remissivo (makeindex)
\makeindex%

%% Arquivo(s) de referências
\addbibresource{fatec-article.bib}

%% Início do documento
\begin{document}

% Seções e subseções
%\section{Título de Seção Primária}%

%\subsection{Título de Seção Secundária}%

%\subsubsection{Título de Seção Terciária}%

%\paragraph{Título de seção quaternária}%

%\subparagraph{Título de seção quinária}%

\section*{Introdução}%
\label{sect:intro}
A ONU (Organização das Nações Unidas) e seus parceiros trabalham para atingir os 17 Objetivos de Desenvolvimento Sustentável (ODS) que abordam os principais desafios enfrentados no Brasil e no mundo \cite{nacoesunidas2024}. Os objetivos envolvem ações para combater a pobreza, proteger o clima e meio ambiente, e garantir paz e prosperidade às pessoas. As ODS 2, ODS 9, ODS 11 e ODS 12 tratam, respectivamente, de fome zero e agricultura sustentável, indústria, inovação e infraestrutura, cidades e comunidades sustentáveis e consumo e produção responsáveis, objetivos que convergem diretamente com o agronegócio.

Atualmente o agronegócio no Mercado Comum do Sul (Mercosul, composto por Argentina, Brasil, Paraguai, Uruguai, Venezuela e Bolívia) é responsável por aproximadamente 10\% das exportações mundiais, sendo o principal exportador de commodities agrícolas básicas \cite{agenciagov2024}. O agronegócio brasileiro representa 22,3\% do PIB (Produto Interno Bruto) em 2024, sendo uma das principais forças econômicas do país \cite{cepea2024}.

O Produto Interno Bruto (PIB) do Estado de São Paulo fechou o ano de 2024 em R\$ 3,5 Tri \cite{agenciasp2025}, representando cerca de 30\% do PIB brasileiro \cite{desiderio2024}. Deste montante, o valor proveniente do PIB da cidade de São Paulo é de 1,12 Tri, ou seja, 32\% do estado \cite{seade2025}. A divisão do PIB municipal mostra que 7,5\% do montante é proveniente da indústria, 20,3\% são impostos líquidos e 72,2\% pertencem ao setor de serviços, não havendo participação do agronegócio \cite{seade2025}. No setor de serviços, destaca-se a quantidade de estabelecimentos voltados à alimentação fora do lar, como padarias, restaurantes e lanchonetes, com 144,9 mil estabelecimentos \cite{sindresbar2024}.

A cidade de São Paulo possui 11,9 milhões de habitantes \cite{agenciabrasil2025}, dos quais 5,8 milhões enfrentam alguma situação de insegurança alimentar \cite{radioagencia2024}. Em contrapartida, estão mapeadas apenas 818 Unidades de Produção Agropecuária e 209 hortas urbanas \cite{sampa2025}.

Fazenda vertical é um conceito criado por Despommier (1999) e vem se aperfeiçoando ao longo dos anos. Consiste em um modelo de cultivo em locais fechados e ambiente controlado, combinado com técnicas como a hidroponia. No Brasil, há mais de 20 fazendas verticais, número baixo em comparação com os EUA, que possuem cerca de 2 mil plantações deste tipo \cite{costa2025}. Para São Paulo, é um modelo aplicável, podendo ampliar a produção em até 30 vezes em um tempo 70\% menor comparado aos modelos tradicionais \cite{gundim2022}.

O cultivo em fazendas verticais apresenta vantagens como ausência de secas, alagamentos e granizo, melhor controle de pragas, economia de água graças ao reuso e não degradação do solo \cite{ingram2023}. É um modelo escalável, podendo ser aplicado desde pequenos espaços até arranha-céus, controlando luz, ambiente, umidade, temperatura, gases e fertirrigação \cite{lucena2021}.

A hidroponia, embora ainda não tão difundida no Brasil, cresce continuamente. Proporciona controle de nutrientes, antecipação da colheita, padronização da quantidade e qualidade, menor incidência de pragas e racionalização da energia. Sua desvantagem é a necessidade de acompanhamento constante do sistema produtivo \cite{luz2006}, altos custos de investimento, dependência energética e necessidade de conhecimento técnico para sua implantação \cite{alves2021}. A alface (Lactuca sativa L.), especialmente o grupo Solta-Crespa, é uma das hortaliças mais importantes, representando 70\% do mercado brasileiro. Combinada com a hidroponia, o tempo de colheita é reduzido em cerca de 10 dias \cite{luz2006}.

A tecnologia IoT (Internet das Coisas) permite conectar objetos inteligentes à internet, transmitindo dados de forma segura \cite{carnaz2016}. Combinada com redes neurais, é útil para monitoramento contínuo em fazendas verticais. O aprendizado de máquina (Machine Learning) é um ramo da IA que constrói sistemas capazes de aprender a partir de dados e gerar modelos de predição ou classificação \cite{paixao2022}. Redes neurais são sistemas compostos por unidades de processamento simples, semelhantes ao funcionamento do cérebro humano \cite{haykin2001}. Entre elas, as redes neurais recorrentes (RNN) tratam dados sequenciais, sendo úteis para reconhecimento de padrões, tradução de textos e predição de valores de mercado \cite{baronte2022}.

Agricultura 4.0 é baseada na aplicação de tecnologias como IoT, robótica, sensores, IA e aprendizado de máquina, visando produtividade, lucratividade e ecoeficiência \cite{lisbinski2020}. Alinhado a esse conceito, temos as Smart Farms, que são fazendas inteligentes equipadas com IoT e conhecimento especializado, permitindo cultivo até por pessoas com pouca experiência e oferecendo vantagens como prevenção e detecção de doenças \cite{ryu2015}.

Na literatura, existem propostas combinando Fazendas Verticais e IoT, com dados utilizados por RNN, permitindo resultados distintos daqueles observados em métodos tradicionais. Uma interação máquina-máquina, combinando IoT e RNN, possibilita a criação de um sistema que trabalha de forma autônoma, aumentando a produtividade com o mínimo de uso de água e terra agricultável.


\section*{OBJETIVO} \label{sect:obj}

O objetivo principal deste projeto é desenvolver uma proposta de ambiente autônomo para gerenciamento de fazendas verticais baseadas em hidroponia, focada em residentes na cidade de São Paulo e região metropolitana. Para alcançar o objetivo geral foram definidos os seguintes objetivos específicos:
1 - Projetar e implementar um sistema web e IoT para o monitoramento e a coleta centralizada de dados de crescimento, nível de água e fertilizantes no cultivo de alface crespa.
2 - Desenvolver e treinar um modelo de Rede Neural capaz de realizar a tomada de decisão autônoma sobre o fluxo de água e fertilizantes.
3 - Propor uma arquitetura de hardware/software de baixo custo, com investimento inicial compatível com o pequeno agricultor urbano da região, e analisar a sua viabilidade econômica em comparação com sistemas comerciais existentes.


\section*{ESTADO DA ARTE} \label{sect:estadoarte}

As fazendas verticais representam um modelo produtivo inovador aos desafios contemporâneos da segurança alimentar, especialmente em centros urbanos densamente povoados. Ao permitir a produção de alimentos em ambientes controlados e em menor espaço físico, essas estruturas contribuem para reduzir a dependência de terras agricultáveis e minimizar os impactos ambientais associados à agricultura tradicional. Entretanto, o pleno aproveitamento desse modelo exige a integração de tecnologias avançadas, como sensores IoT, inteligência artificial e redes neurais, capazes de monitorar continuamente variáveis críticas — como umidade, luminosidade, temperatura e concentração de nutrientes — e realizar ajustes automáticos em tempo real. Dessa forma, não apenas se garante maior eficiência e produtividade, como também se promove um manejo sustentável, acessível e alinhado às metas globais de desenvolvimento sustentável. Realizando busca por artigos neste sentido, foram identificados alguns trabalhos com a mesma temática. Saraswathy et al \citeyear{saraswathy2020} apresentaram um trabalho muito parecido com este, com o objetivo de integrar inteligência artificial utilizando IoT em uma fazenda hidropônica focada no estado de Tamil Nadu, no sul da Índia. A metodologia proposta monitorava parâmetros de umidade, pH, temperatura, intensidade de luz e fluxo de água por meio de sensores, enviando os valores para a nuvem via Node MCU. Utilizando uma rede neural recorrente (RNN) do tipo Long-Short-Term Memory (LSTM) com algoritmo de previsão visando maior precisão na automação, o trabalho apresentou resultados satisfatórios, eliminando a necessidade de acompanhamento constante pelos fazendeiros e permitindo que os erros apresentados pela rede neural fossem utilizados para automatizar toda a produção da fazenda hidropônica.

Sousa \citeyear{sousa2023} desenvolveram um projeto sobre um sistema baseado em IoT e sensores com o objetivo de supervisionar e controlar de fazendas verticais. A metodologia era baseada em um módulo gerenciado por microcontrolador, estufa e banco de dados hospedado em um computador pessoal. O módulo dispunha de entradas e saídas analógicas e digitais (para conexão dos sensores utilizados), além de conexão sem-fio. O autor relatou que os objetivos propostos foram alcançados, utilizando um equipamento de custo acessível e monitoramento constante.

Rakhmatulin et al \citeyear{rakhmatulin2021} apresentram um sistema automatizado combinando IoT e rede neural, juntamente com um software desenvolvido pelo autor, câmera digital e sensores de luminosidade, gás carbônico, temperatura, umidade, ph e temperatura do solo. Em sua metodologia o autor se utilizou de um ambiente fechado em que 7 sensores monitoram o crescimento da hortaliça. Por sua vez o sinal dos sensores é encaminhado para um controlador que envia os dados para uma rede neural avaliar e fazer a gestão do cultivo.

Ahmareen et al \citeyear{ahmareen2024} apresentaram um sistema utilizando IoT para gerenciar uma fazenda vertical. Combinado com tecnologias para monitoramento remoto, sensores e bancos de dados, o sistema buscava melhorar a gestão da fazenda vertical e desenvolver um modelo de custo acessível. A metodologia utilizada combinava sensores de umidade, chuva, solo e temperatura com um controlador que, por sua vez, encaminhava os dados para a nuvem. Estes dados eram convertidos em gráficos e apresentados em uma tela, tornando possível a tomada de decisão no controle de uma bomba d'água e sistema de ventilação.

Embora os trabalhos de Saraswathy \citeyear{saraswathy2020} e Rakhmatulin et al \citeyear{rakhmatulin2021} empreguem Redes Neurais, eles não focam na agricultura hidropônica vertical ou em um custo acessível. Os trabalhos de Sousa \citeyear{sousa2023} e Ahmareen et al \citeyear{ahmareen2024} focam no baixo custo, mas carecem da autonomia proporcionada pela IA. A contribuição central deste trabalho reside em combinar a autonomia da RN com uma arquitetura de baixo custo, valendo-se de um escopo reduzido (apenas água e fertilizante) para a realidade da agricultura urbana em São Paulo.

\section*{METODOLOGIA} \label{sect:metodologia}

Para a problemática de uma fazenda vertical orientada por redes neurais foi definida a necessidade de mapear uma solução lógica de como o sistema de gerenciamento irá funcionar. O mapeamento (a partir da literatura existente) indicou duas possibilidades: no Fluxograma 1 a fazenda vertical (1) envia, por meio de sensores, os dados atuais de fluxo de água e fertilização para um banco de dados na nuvem (2). Estes dados podem ser acessados pelo cliente, por meio de software, onde o mesmo pode tomar a decisão como como devem operar o fluxo de água e fertilização da fazenda, recomeçando o ciclo.
\begin{figure}[h!]
    \centering
    \includegraphics[scale=0.2]{Illustrations/Fluxograma1.png} % sem extensão se for .png
    \caption{Exemplo de funcionamento do sistema proposto}
    \label{fcht:fluxograma1}
    % \SourceOrNote{Autoria Própria (2024)} % só se esse comando estiver definido
\end{figure}

O Fluxograma 2 exemplifica que, inicialmente, a fazenda vertical (1) envia os dados de fertilizantes e fluxo de agua (2) para um banco de dados na nuvem (3). Porém quem faz a avaliação e tomada de decisão com base nos dados é uma Rede Neural (4), recomeçando o ciclo. Assim o cliente apenas participa do ciclo caso queira (seguindo assim o fluxograma 1).

\begin{figure}[h!]
    \centering
    \includegraphics[scale=0.2]{Illustrations/Fluxograma2.png} % sem extensão se for .png
    \caption{Exemplo de funcionamento do sistema proposto}
    \label{fcht:fluxograma2}
    % \SourceOrNote{Autoria Própria (2024)} % só se esse comando estiver definido
\end{figure}


A seguir, são apresentadas as etapas planejadas do desenvolvimento do trabalho:

Revisão de Estudos \\
Realizado levantamento dos trabalhos mais atuais referentes a fazendas verticais, IoT, fazendas neurais e tecnologias semelhantes.

Planejamento de estrutura e definição dos componentes \\
Definição do tipo de tecnologia, armazenamento do banco de dados, layout do sistema e funcionalidades.

Construção de modelo em pequena escala \\
Elaboração de um pequeno protótipo para avaliação prática do projeto.

Implementação do sistema \\
Desenvolvimento do sistema, melhorias, testes e correções.

\section*{RESULTADOS PRELIMINARES}\label{sect:resultados}

\input{Topicos/resultados}

\section*{CONCLUSÃO}\label{sect:conclusao}

\input{Topicos/conclusao}

\printbibliography

%% Elementos pós-textuais (opcionais): Apêndice e Anexo
%Caso for utilizar, basta retirar o símbolo de % na frente do comando
%\input{./Extras/post-textual}

%% Fim do documento
\end{document}