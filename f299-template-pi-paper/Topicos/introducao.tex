A ONU (Organização das Nações Unidas) e seus parceiros trabalham para atingir os 17 Objetivos de Desenvolvimento Sustentável (ODS) que abordam os principais desafios enfrentados no Brasil e no mundo \cite{nacoesunidas2024}. Os objetivos envolvem ações para combater a pobreza, proteger o clima e meio ambiente, e garantir paz e prosperidade às pessoas. As ODS 2, ODS 9, ODS 11 e ODS 12 tratam, respectivamente, de fome zero e agricultura sustentável, indústria, inovação e infraestrutura, cidades e comunidades sustentáveis e consumo e produção responsáveis, objetivos que convergem diretamente com o agronegócio.

Atualmente o agronegócio no Mercado Comum do Sul (Mercosul, composto por Argentina, Brasil, Paraguai, Uruguai, Venezuela e Bolívia) é responsável por aproximadamente 10\% das exportações mundiais, sendo o principal exportador de commodities agrícolas básicas \cite{agenciagov2024}. O agronegócio brasileiro representa 22,3\% do PIB (Produto Interno Bruto) em 2024, sendo uma das principais forças econômicas do país \cite{cepea2024}.

O Produto Interno Bruto (PIB) do Estado de São Paulo fechou o ano de 2024 em R\$ 3,5 Tri \cite{agenciasp2025}, representando cerca de 30\% do PIB brasileiro \cite{desiderio2024}. Deste montante, o valor proveniente do PIB da cidade de São Paulo é de 1,12 Tri, ou seja, 32\% do estado \cite{seade2025}. A divisão do PIB municipal mostra que 7,5\% do montante é proveniente da indústria, 20,3\% são impostos líquidos e 72,2\% pertencem ao setor de serviços, não havendo participação do agronegócio \cite{seade2025}. No setor de serviços, destaca-se a quantidade de estabelecimentos voltados à alimentação fora do lar, como padarias, restaurantes e lanchonetes, com 144,9 mil estabelecimentos \cite{sindresbar2024}.

A cidade de São Paulo possui 11,9 milhões de habitantes \cite{agenciabrasil2025}, dos quais 5,8 milhões enfrentam alguma situação de insegurança alimentar \cite{radioagencia2024}. Em contrapartida, estão mapeadas apenas 818 Unidades de Produção Agropecuária e 209 hortas urbanas \cite{sampa2025}.

Fazenda vertical é um conceito criado por Despommier (1999) e vem se aperfeiçoando ao longo dos anos. Consiste em um modelo de cultivo em locais fechados e ambiente controlado, combinado com técnicas como a hidroponia. No Brasil, há mais de 20 fazendas verticais, número baixo em comparação com os EUA, que possuem cerca de 2 mil plantações deste tipo \cite{costa2025}. Para São Paulo, é um modelo aplicável, podendo ampliar a produção em até 30 vezes em um tempo 70\% menor comparado aos modelos tradicionais \cite{gundim2022}.

O cultivo em fazendas verticais apresenta vantagens como ausência de secas, alagamentos e granizo, melhor controle de pragas, economia de água graças ao reuso e não degradação do solo \cite{ingram2023}. É um modelo escalável, podendo ser aplicado desde pequenos espaços até arranha-céus, controlando luz, ambiente, umidade, temperatura, gases e fertirrigação \cite{lucena2021}.

A hidroponia, embora ainda não tão difundida no Brasil, cresce continuamente. Proporciona controle de nutrientes, antecipação da colheita, padronização da quantidade e qualidade, menor incidência de pragas e racionalização da energia. Sua desvantagem é a necessidade de acompanhamento constante do sistema produtivo \cite{luz2006}. A alface (Lactuca sativa L.), especialmente o grupo Solta-Crespa, é uma das hortaliças mais importantes, representando 70\% do mercado brasileiro. Combinada com a hidroponia, o tempo de colheita é reduzido em cerca de 10 dias \cite{luz2006}.

A tecnologia IoT (Internet das Coisas) permite conectar objetos inteligentes à internet, transmitindo dados de forma segura \cite{carnaz2016}. Combinada com redes neurais, é útil para monitoramento contínuo em fazendas verticais. O aprendizado de máquina (Machine Learning) é um ramo da IA que constrói sistemas capazes de aprender a partir de dados e gerar modelos de predição ou classificação \cite{paixao2022}. Redes neurais são sistemas compostos por unidades de processamento simples, semelhantes ao funcionamento do cérebro humano \cite{haykin2001}. Entre elas, as redes neurais recorrentes (RNN) tratam dados sequenciais, sendo úteis para reconhecimento de padrões, tradução de textos e predição de valores de mercado \cite{baronte2022}.

Agricultura 4.0 é baseada na aplicação de tecnologias como IoT, robótica, sensores, IA e aprendizado de máquina, visando produtividade, lucratividade e ecoeficiência \cite{lisbinski2020}. Alinhado a esse conceito, temos as Smart Farms, que são fazendas inteligentes equipadas com IoT e conhecimento especializado, permitindo cultivo até por pessoas com pouca experiência e oferecendo vantagens como prevenção e detecção de doenças \cite{ryu2015}.

Na literatura, existem propostas combinando Fazendas Verticais e IoT, com dados utilizados por RNN, permitindo resultados distintos daqueles observados em métodos tradicionais.