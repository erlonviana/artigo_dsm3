A cidade de São Paulo possui 11,9 milhões de habitantes \cite{agenciabrasil2025}, dos quais 5,8 milhões enfrentam alguma situação de insegurança alimentar \cite{radioagencia2024}. Em contrapartida, estão mapeadas apenas 818 Unidades de Produção Agropecuária e 209 hortas urbanas \cite{sampa2025}. Neste contexto, a proposta de fazendas verticais é uma alternativa.

Fazenda vertical é um conceito criado por Despommier (1999) e vem se aperfeiçoando ao longo dos anos. Consiste em um modelo de cultivo em locais fechados e ambiente controlado, combinado com técnicas como a hidroponia. No Brasil, há mais de 20 fazendas verticais, número baixo em comparação com os EUA, que possuem cerca de 2 mil plantações deste tipo \cite{costa2025}. Para São Paulo, é um modelo aplicável, podendo ampliar a produção em até 30 vezes em um tempo 70\% menor comparado aos modelos tradicionais \cite{gundim2022}.

O cultivo em fazendas verticais apresenta vantagens como ausência de secas, alagamentos e granizo, melhor controle de pragas, economia de água graças ao reuso e não degradação do solo \cite{ingram2023}. É um modelo escalável, podendo ser aplicado desde pequenos espaços até arranhacéus, controlando luz, ambiente, umidade, temperatura, gases e fertirrigação \cite{lucena2021}. Entre os métodos de cultivo utilizados temos a hidroponia.

Hidroponia, embora ainda não tão difundida no Brasil, cresce continuamente. Proporciona controle de nutrientes, antecipação da colheita, padronização da quantidade e qualidade, menor incidência de pragas e racionalização da energia. Sua desvantagem é a necessidade de acompanhamento constante do sistema produtivo \cite{luz2006}, altos custos de investimento, dependência energética e necessidade de conhecimento técnico para sua implantação \cite{alves2021}. A alface (Lactuca sativa L.), especialmente o grupo Solta-Crespa, é uma das hortaliças mais importantes, representando 70\% do mercado brasileiro. Combinada com a hidroponia, o tempo de colheita é reduzido em cerca de 10 dias \cite{luz2006}. Além disso, podemos combinar a hidroponia com a tecnologia de Internet das Coisas.

A tecnologia Internet das Coisas (do inglês, Internet of the Things ou IoT) permite conectar objetos inteligentes à internet, transmitindo dados de forma segura \cite{carnaz2016}. O aprendizado de máquina (Machine Learning) é um ramo da Inteligência Artificial que constrói sistemas capazes de aprender a partir de dados e gerar modelos de predição ou classificação \cite{paixao2022}. Já as redes neurais são sistemas compostos por unidades de processamento simples, semelhantes ao funcionamento do cérebro humano \cite{haykin2001}. Entre elas, as redes neurais recorrentes (RNN) tratam dados sequenciais, sendo úteis para reconhecimento de padrões, tradução de textos e predição de valores de mercado \cite{baronte2022}. Tanto IoT quanto Redes Neurais são ferramentas cada vez mais presentes na Agricultura 4.0.

Por sua vez, Agricultura 4.0 é baseada na aplicação de tecnologias como IoT, robótica, sensores, IA e aprendizado de máquina, visando produtividade, lucratividade e ecoeficiência \cite{lisbinski2020}. Alinhado a esse conceito, temos as Smart Farms, que são fazendas inteligentes equipadas com IoT e conhecimento especializado, permitindo cultivo até por pessoas com pouca experiência e oferecendo vantagens como prevenção e detecção de doenças \cite{ryu2015}.

Esta é uma proposta de interação máquina-máquina, combinando IoT e RNN, criando assim um sistema que trabalhe de forma autônoma, visando um sistema com valor acessível, produtividade com o mínimo de uso de água e terra agricultável, em consonância com os Objetivos de Desenvolvimento Sustentável (ODS 2, ODS 9, ODS 11 e ODS 12, respectivamente, fome zero e agricultura sustentável, indústria, inovação e infraestrutura, cidades e comunidades sustentáveis e consumo e produção responsáveis) da Organização das Nações Unidas \cite{nacoesunidas2024}.

Desta forma, buscamos responder o questionamento: como utilizar redes neurais em conjunto com IoT para tornar fazendas verticais mais autônomas, reduzindo custos operacionais, aumentando a eficiência no uso de recursos naturais e mantendo um valor acessível?