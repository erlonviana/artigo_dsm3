As fazendas verticais representam um modelo produtivo inovador aos desafios contemporâneos da segurança alimentar, especialmente em centros urbanos densamente povoados. Ao permitir a produção de alimentos em ambientes controlados e em menor espaço físico, essas estruturas contribuem para reduzir a dependência de terras agricultáveis e minimizar os impactos ambientais associados à agricultura tradicional. Entretanto, o pleno aproveitamento desse modelo exige a integração de tecnologias avançadas, como sensores IoT, inteligência artificial e redes neurais, capazes de monitorar continuamente variáveis críticas — como umidade, luminosidade, temperatura e concentração de nutrientes — e realizar ajustes automáticos em tempo real. Dessa forma, não apenas se garante maior eficiência e produtividade, como também se promove um manejo sustentável, acessível e alinhado às metas globais de desenvolvimento sustentável. Realizando busca por artigos neste sentido, foram identificados alguns trabalhos com a mesma temática. Saraswathy et al \citeyear{saraswathy2020} apresentaram um trabalho muito parecido com este, com o objetivo de integrar inteligência artificial utilizando IoT em uma fazenda hidropônica focada no estado de Tamil Nadu, no sul da Índia. A metodologia proposta monitorava parâmetros de umidade, pH, temperatura, intensidade de luz e fluxo de água por meio de sensores, enviando os valores para a nuvem via Node MCU. Utilizando uma rede neural recorrente (RNN) do tipo Long-Short-Term Memory (LSTM) com algoritmo de previsão visando maior precisão na automação, o trabalho apresentou resultados satisfatórios, eliminando a necessidade de acompanhamento constante pelos fazendeiros e permitindo que os erros apresentados pela rede neural fossem utilizados para automatizar toda a produção da fazenda hidropônica.

Sousa \citeyear{sousa2023} desenvolveram um projeto sobre um sistema baseado em IoT e sensores com o objetivo de supervisionar e controlar de fazendas verticais. A metodologia era baseada em um módulo gerenciado por microcontrolador, estufa e banco de dados hospedado em um computador pessoal. O módulo dispunha de entradas e saídas analógicas e digitais (para conexão dos sensores utilizados), além de conexão sem-fio. O autor relatou que os objetivos propostos foram alcançados, utilizando um equipamento de custo acessível e monitoramento constante.

Rakhmatulin et al \citeyear{rakhmatulin2021} apresentram um sistema automatizado combinando IoT e rede neural, juntamente com um software desenvolvido pelo autor, câmera digital e sensores de luminosidade, gás carbônico, temperatura, umidade, ph e temperatura do solo. Em sua metodologia o autor se utilizou de um ambiente fechado em que 7 sensores monitoram o crescimento da hortaliça. Por sua vez o sinal dos sensores é encaminhado para um controlador que envia os dados para uma rede neural avaliar e fazer a gestão do cultivo.

Ahmareen et al \citeyear{ahmareen2024} apresentaram um sistema utilizando IoT para gerenciar uma fazenda vertical. Combinado com tecnologias para monitoramento remoto, sensores e bancos de dados, o sistema buscava melhorar a gestão da fazenda vertical e desenvolver um modelo de custo acessível. A metodologia utilizada combinava sensores de umidade, chuva, solo e temperatura com um controlador que, por sua vez, encaminhava os dados para a nuvem. Estes dados eram convertidos em gráficos e apresentados em uma tela, tornando possível a tomada de decisão no controle de uma bomba d'água e sistema de ventilação.

Embora os trabalhos de Saraswathy \citeyear{saraswathy2020} e Rakhmatulin et al \citeyear{rakhmatulin2021} empreguem Redes Neurais, eles não focam na agricultura hidropônica vertical ou em um custo acessível. Os trabalhos de Sousa \citeyear{sousa2023} e Ahmareen et al \citeyear{ahmareen2024} focam no baixo custo, mas carecem da autonomia proporcionada pela IA. A contribuição central deste trabalho reside em combinar a autonomia da RN com uma arquitetura de baixo custo, valendo-se de um escopo reduzido (apenas água e fertilizante) para a realidade da agricultura urbana em São Paulo.