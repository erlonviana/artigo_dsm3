O objeto de estudo deste projeto são fazendas verticais e o uso de tecnologia. Realizando busca por artigos neste sentido, foram identificados alguns trabalhos com a mesma temática. Saraswathy et al. \cite{saraswathy2020} apresentaram um trabalho muito parecido com o nosso, relatando a integração de inteligência artificial utilizando IoT em uma fazenda hidropônica focada no estado de Tamil Nadu, no sul da Índia. A proposta monitorava parâmetros de umidade, pH, temperatura, intensidade de luz e fluxo de água por meio de sensores, enviando os valores para a nuvem via Node MCU. Utilizando uma rede neural recorrente (RNN) do tipo Long-Short-Term Memory (LSTM) com algoritmo de previsão visando maior precisão na automação, o trabalho apresentou resultados satisfatórios, eliminando a necessidade de acompanhamento constante pelos fazendeiros e permitindo que os erros apresentados pela rede neural fossem utilizados para automatizar toda a produção da fazenda hidropônica. Este trabalho se assemelha ao nosso pelo uso de redes neurais em um ambiente de fazenda vertical, embora não utilize aplicação móvel e monitore mais parâmetros do que os planejados neste estudo.

Souza \citeyear{sousa2023} desenvolveu um projeto sobre um sistema baseado em IoT e sensores para supervisão e controle de fazendas verticais. O projeto foi constituído de um módulo gerenciado por microcontrolador, estufa e banco de dados hospedado em um computador pessoal. O módulo dispunha de entradas e saídas analógicas e digitais (para conexão dos sensores utilizados), além de conexão sem-fio. O autor relatou que os objetivos propostos foram alcançados, utilizando um equipamento de custo acessível e monitoramento constante. O trabalho se assemelha à nossa proposta pelo uso de fazendas verticais e IoT, porém diverge por não utilizar nenhum sistema de inteligência artificial, tratando os dados coletados apenas dentro de intervalos para tomada de decisão.

Rakhmatulin \cite{rakhmatulin2021} apresentou um sistema automatizado combinando IoT e rede neural, juntamente com um software desenvolvido pelo autor, câmera digital e sensores de luminosidade, gás carbônico, temperatura, umidade, ph e temperatura do solo. Em sua proposta o autor se utilizou de um ambiente fechado em que 7 sensores monitoram o crescimento da hortaliça. Por sua vez o sinal dos sensores é encaminhado para um controlador que envia os dados para uma rede neural avaliar e fazer a gestão do cultivo. Este sistema difere da nossa proposta por empregar agricultura convencional e ter foco no cultivo de tomates.

Ahmareen et al. \cite{ahmareen2024} apresentaram um sistema utilizando IoT para gerenciar uma fazenda vertical. Combinado com tecnologias para monitoramento remoto, sensores e bancos de dados, o sistema buscava melhorar a gestão da fazenda vertical e desenvolver um modelo de custo acessível. A metodologia utilizada combinava sensores de umidade, chuva, solo e temperatura com um controlador que, por sua vez, encaminhava os dados para a nuvem. Estes dados eram convertidos em gráficos e apresentados em uma tela, tornando possível a tomada de decisão no controle de uma bomba d’água e sistema de ventilação. Embora os objetivos tenham sido alcançados, o autor encontrou dificuldades com os custos de instalação e manutenção do sistema IoT, além da instabilidade da internet em algumas localidades. Diferentemente do nosso trabalho, não foi utilizado nenhum sistema de rede neural para gerenciamento automático.
